\documentclass[openany]{book}

\usepackage[utf8]{inputenc}
\usepackage[italian]{babel}
\usepackage[T1]{fontenc}
\usepackage{natbib}
\usepackage{amsmath}
\usepackage{latexsym}
\usepackage{geometry}
\usepackage{amsfonts}
\usepackage{blindtext} % \blindtext \Blindtext \Blinddocument

 \geometry{
 a4paper,
 left=25mm,
 right=25mm,
 %top=25mm,
 bottom=30mm,
}

\title{\Huge \texttt{Metodi Numerici}}
\date{}
\author{\textsf{MT}}

\renewcommand{\baselinestretch}{1.1}

\setlength{\parindent}{0em}
\setlength{\parskip}{0.2em}

\begin{document}

\maketitle

\tableofcontents

\chapter{Sistemi lineari}

\section{Fattorizzazioni di matrici}

\subsection{Fattorizzazione di Gauss}

Successione finita di trasformazioni della matrice dei coefficienti $A$ e del termine noto $b$, cioè moltiplicazione per un numero finito di opportune matrici

Per scambiare le righe $i$ e $j$, si moltiplica A per la matrice identità con le righe $ i $  e $ j$  scambiate

Per sostituire la riga $ i$ con la stessa più la $ j$ moltiplicata per $m_{ij}$, si usa $ I + m_{ij}e_ie_j^T$

Combiniamo queste matrici in modo che il nuovo sistema assuma forma triangolare superiore $ Ux = \bar{b} $ ponendo $ GA = U $

Costo: $ n^3/3$ operazioni aritmetiche 

È inutile memorizzare G, ma è sufficiente memorizzare i moltiplicatori $ m_{ij}$ (nella parte inferiore della matrice $A$) e le permutazioni effettuate 

Anche costruire $P_i$ è superfluo, basta avere un vettore di pivot e se al passo k-esimo viene effettuata la permutazione tra $k$ e $j$, basta porre $\text{pivot}(k)=j$ 
\\

Il metodo di Gauss può essere anche utilizzato per determinare: 

\begin{itemize}
	\item matrice di permutazione $P$
	\item matrice triangolare inferiore con diagonale unitaria  $L$
	\item matrice triangolare superiore $U$
\end{itemize}

tali che $PA=LU $

N.B. tale decomposizione per una matrice qualsisi non singolare non è unica!

\begin{itemize}
	\item se $A$ è simmetrica e a diagonale dominante, il pivoting parziale non produce scambi
	\item se $ A$ è simmetrica definita positiva, l'algoritmo è stabile anche senza pivoting
\end{itemize}

quindi otteniamo $A=LU$

\subsection {Fattorizzazione di Choleski}

Se $A$ è \textit{simmetrica} e \textit{definita positiva}, abbiamo $A=LU$, che possiamo riscrivere come $A=LDU_1$, dove $U_1$ si ottiene da $U$ dividendo ogni riga per il suo elemento diagonale che viene messo in $D$

\begin{itemize}
	\item Poiché $A$ è \textit{simmetrica}, $U_1 = L^T \Rightarrow A=LDL^T$ 
	\item Poiché $A$ è \textit{definita positiva} ($\Leftrightarrow u_{ii}>0 \quad \forall i$), definiamo $D^{1/2}=\text {diag}(\sqrt {u_{ii}})\Rightarrow D=D^{1/2}D^{1/2}$

\end{itemize}

$\Rightarrow A=LD^{1/2}(D^{1/2})^T L^T\Rightarrow A=(LD^{1/2})(LD^{1/2})^T=L_1L_1^T$
\\

$L_1$ è una matrice triangolare inferiore con elementi diagonali positivi

Costo: $n^3/6$ operazioni

\subsection {Fattorizzazione QR}

\subsubsection {Riflettori elementari}

Dato un vettore $x=(x_1,x_2,...,x_n)^T$ è possibile determinare un \textit{riflettore elementare} $U_k \equiv U_k^{(n)}$ tale che $U_kx=(\bar{x}_1,...,\bar{x}_k,0,...,0)^T$

\end{document}





































