\documentclass[openany]{book}

\usepackage[utf8]{inputenc}
\usepackage[italian]{babel}
\usepackage[T1]{fontenc}
\usepackage{natbib}
\usepackage{amsmath}
\usepackage{latexsym}
\usepackage{geometry}
\usepackage{amsfonts}
\usepackage{blindtext} % \blindtext \Blindtext \Blinddocument

 \geometry{
 a4paper,
 left=25mm,
 right=25mm,
 %top=25mm,
 bottom=30mm,
}

\title{\Huge \texttt{Metodi Numerici}}
\date{}
\author{\textsf{MT}}

\renewcommand{\baselinestretch}{1.1}

\setlength{\parindent}{0em}
\setlength{\parskip}{0.2em}

\usepackage{hyperref}
\hypersetup{
    colorlinks=true,
    linkcolor=blue,
    filecolor=magenta,      
    urlcolor=cyan,
    pdftitle={Metodi Numerici},
    bookmarks=true,
    %pdfpagemode=FullScreen,
}


\begin{document}

\maketitle

\tableofcontents

\chapter{Sistemi lineari}

\section{Fattorizzazioni di matrici}
%%%%%%%%%%%%%%%%%%%%%%%%%%%%%%%%%%%%%
\subsection{Fattorizzazione di Gauss}

Successione finita di trasformazioni della matrice dei coefficienti $A$ e del termine noto $b$, cioè 
moltiplicazione per un numero finito di opportune matrici

Per scambiare le righe $i$ e $j$, si moltiplica A per la matrice identità con le righe $ i $  e $ j$  scambiate

Per sostituire la riga $ i$ con la stessa più la $ j$ moltiplicata per $m_{ij}$, si usa $ I + m_{ij}e_ie_j^T$

Combiniamo queste matrici in modo che il nuovo sistema assuma forma triangolare superiore $ Ux = \bar{b} $ 
ponendo $ GA = U $

Costo: $ n^3/3$ operazioni aritmetiche

È inutile memorizzare G, ma è sufficiente memorizzare i moltiplicatori $ m_{ij}$ (nella parte inferiore della 
matrice $A$) e le permutazioni effettuate

Anche costruire $P_i$ è superfluo, basta avere un vettore di pivot e se al passo k-esimo viene effettuata la 
permutazione tra $k$ e $j$, basta porre $\text{pivot}(k)=j$
%%%%%%%%%%%%%%%%%%%%%%%%%%%%%%%%%
\subsection{Fattorizzazione $LU$}

Il metodo di Gauss può essere anche utilizzato per determinare: 

\begin{itemize}
	\item matrice di permutazione $P$
	\item matrice triangolare inferiore con diagonale unitaria  $L$
	\item matrice triangolare superiore $U$
\end{itemize}

tali che $PA=LU $

N.B. tale decomposizione per una matrice qualsisi non singolare non è unica!

\begin{itemize}
	\item se $A$ è simmetrica e a diagonale dominante, il pivoting parziale non produce scambi
	\item se $ A$ è simmetrica definita positiva, l'algoritmo è stabile anche senza pivoting
\end{itemize}

quindi otteniamo $A=LU$
%%%%%%%%%%%%%%%%%%%%%%%%%%%%%%%%%%%%%%%%%
\subsection {Fattorizzazione di Choleski}

Se $A$ è \textit{simmetrica} e \textit{definita positiva}, abbiamo $A=LU$, che possiamo riscrivere come 
$A=LDU_1$, dove $U_1$ si ottiene da $U$ dividendo ogni riga per il suo elemento diagonale che viene messo in $D$

\begin{itemize}
	\item Poiché $A$ è \textit{simmetrica}, $U_1 = L^T \Rightarrow A=LDL^T$ 
	\item Poiché $A$ è \textit{definita positiva} ($\Leftrightarrow u_{ii}>0 \quad \forall i$), definiamo 
	$D^{1/2}=\text {diag}(\sqrt {u_{ii}})\Rightarrow D=D^{1/2}D^{1/2}$

\end{itemize}

$\Rightarrow A=LD^{1/2}(D^{1/2})^T L^T\Rightarrow A=(LD^{1/2})(LD^{1/2})^T=L_1L_1^T$
\\

$L_1$ è una matrice triangolare inferiore con elementi diagonali positivi

Costo: $n^3/6$ operazioni
%%%%%%%%%%%%%%%%%%%%%%%%%%%%%%%%
\subsection {Fattorizzazione $QR$}

\subsubsection {Riflettori elementari}

Dato un vettore $x=(x_1,x_2,...,x_n)^T$ è possibile determinare un \textit{riflettore elementare} $U_k \equiv 
U_k^{(n)}$ tale che $U_kx=(\bar{x}_1,...,\bar{x}_k,0,...,0)^T$
\\

\quad {\large $U_k = I - \frac{1}{\pi_k}u_ku_k^T$ }
\\

con \quad $u_k=\left( \begin{array}{c} o \\ x_k + \sigma_k \\ x_{k+1} \\ ... \\ x_n \end{array} \right)$ 
\quad , \quad 
 $o\in \mathbb{R}^{k-1}$ \quad , \quad
$\sigma_k = \pm ||x_k'||_2$ \quad , \quad 
$x_k'= \left( \begin{array}{c} x_k \\ x_{k+1} \\ ... \\ x_n \end{array} \right)$ \quad , \quad 
$\pi_k=\frac{1}{2}||u_k||_2^2$ 

Per $\sigma_k$, scegliere segno concorde per non avere cancellazione numerica

\subsubsection{QR}


Con l'utilizzo dei riflettori possiamo triangolarizzare la matrice, da cui $Q^TA=R \Rightarrow A=QR$, dove:

\begin{itemize}
	\item essendo le $U_k$ ortogonali, $Q$ è ortogonale
	\item $R$ è triangolare superiore
\end{itemize}

Fattorizzazione perfettamente stabile, ma onerosa, perciò si usa quando Gauss fallisce ($\frac{2n^3}{3}$ vs 
$\frac{n^3}{3}$)

Anche in questo caso $Q$ non viene definita esplicitamente ma si salvano solo $u_k$ e $\pi_k$

Si può anche applicare alle matrici rettangolari

\newpage

\section{Applicazioni}
%%%%%%%%%%%%%%%%%%%%%%%%%%%%%%%%%%%%%%%%%%%%%%%%
\subsection{Soluzione del sistema lineare $Ax=b$}

$PAx=Pb \Rightarrow LUx=b' \Rightarrow \left\{ \begin{array}{ll} Ly=b' & \Rightarrow y \\ Ux = y & \Rightarrow 
\boldsymbol{x} \end{array}\right.$ \qquad costo: $\frac{n^3}{6}$
\\

Se invece $A$ è \textit{simmetrica} e \textit{definita positiva}

$Ax=b \Rightarrow LL^Tx=b \Rightarrow \left\{ \begin{array}{ll} Ly=b' & \Rightarrow y \\ L^Tx = y & \Rightarrow
 \boldsymbol{x} \end{array}\right.$ \qquad costo: $\frac{n^3}{6}$
\\

Utilizzando la \textit{fattorizzazione $QR$}

$Q^TAx=Q^Tb \Rightarrow Rx=b'$ \qquad costo: $\frac{2n^3}{6}$
%%%%%%%%%%%%%%%%%%%%%%%%%%%%%%%%%%%%%%%%%%
\subsection{Calcolo della matrice inversa}

L'inversione di una matrice triangolare coinvolge $\frac{n^3}{6}$ operazioni aritmetiche, mentre quella di una 
matrice generica è $n^3$. La moltiplicazione tra due matrici richiede invece $n^3$ operazioni, mentre la 
risoluzione di un sistema lineare $n^2$

N.B. l'inversa di una matrice sparsa è generalmente densa

\begin{itemize}
	\item $\boldsymbol{GA=U}\Rightarrow (GA)^{-1}=U^{-1}\Rightarrow A^{-1}G^{-1}=U^{-1}\Rightarrow 
	\boldsymbol{A^{-1}=U^{-1}G}$
	\item $\boldsymbol{PA=LU} \Rightarrow (PA)^{-1}=(LU)^{-1}\Rightarrow A^{-1}P^{-1}=U^{-1}L^{-1}\Rightarrow 
	\boldsymbol{A^{-1}=U^{-1}L^{-1}P}$
	\item $\boldsymbol{A=LL^T}\Rightarrow A^{-1}(LL^T)^{-1}\Rightarrow A^{-1}=(L^T)^{-1}L^{-1} \Rightarrow 
	\boldsymbol{A^{-1}=(L^{-1})^TL^{-1}}$ \quad L'inversa della trasposta è la trasposta dell'inversa 
	\item $\boldsymbol{A=QR}\Rightarrow A^{-1}=(QR)^{-1}\Rightarrow A^{-1}=R^{-1}Q^{-1} \Rightarrow 
	\boldsymbol{A^{-1}=R^{-1}Q^T}$ \quad $Q$ è ortogonale quindi l'inversa coincide con la trasposta
\end{itemize}
%%%%%%%%%%%%%%%%%%%%%%%%%%%%%%%%%%%%%%%%%%%%%%%%%%%%%%%%%%%%%%%%%%%%%%%%%%%%%%%%%%%%%%%%%%%%%%%%%%%
\subsection{Calcolare la soluzione del sistema $(A^TA)x=b$ senza costruire esplicitamente $A^TA$}

\subsubsection{Fattorizzazione $LU$}

$A^TAx=b \Rightarrow \left\{ \begin{array}{ll} A^Ty=b & \Rightarrow y \\ Ax=y & \Rightarrow \boldsymbol{x} 
\end{array} \right.$ 

$PA=LU \Rightarrow (PA)^T=(LU)^T \Rightarrow A^TP^T=U^TL^T \Rightarrow \boldsymbol{A^T=U^TL^TP}$ \quad N.B. 
$P$ è \textit{ortogonale}, ma non simmetrica

$A^Ty=b \Rightarrow U^TL^TPy=b \Rightarrow \left\{ \begin{array}{ll} U^T\alpha = b & \Rightarrow \alpha \\ 
L^T\beta = \alpha & \Rightarrow \beta \end{array} \right.$ 

$Py=\beta \Rightarrow y=P^T\beta$ e si sostituisce nel sistema iniziale 

\subsubsection{Fattorizzazione $QR$}

$A=QR \Rightarrow A^TA = R^TQ^TQR=R^TR$ \qquad $Q^TQ=I$ perché $Q$ è ortogonale
\\
		
$Ax=b \Rightarrow R^TRx=b \Rightarrow \left\{ \begin{array}{ll} R^Ty=b & \Rightarrow y \\ Rx=y & \Rightarrow 
\boldsymbol{x} \end{array} \right.$

\newpage
%%%%%%%%%%%%%%%%%%%%%%%%%%%
\section{Metodi iterativi}

In alcune situazioni i sistemi da risolvere sono sparsi e molto grandi e il metodo di Gauss rischia di rendere 
le matrici sempre meno sparse portando alla memorizzazione di un numero eccessivo di valori.
I metodi iterativi hanno il vantaggio di non alterare mai la matrice iniziale.
Partendo da una approssimazione iniziale della soluzione, e utilizzando sempre solo gli elementi non nulli 
della matrice, si crea una successione di soluzioni approssimate convergente alla soluzione esatta del sistema 
non singolare $Ax=b$
\\

È utile decomporre la matrice $A$ nella forma $A=D+C$ e riscrivere il sistema come $Dx=-Cx+b$. Partendo da un 
generico vettore $x^{(0)}$ costruiamo la successione $x^{(1)},x^{(2)},...$ innescando il processo iterativo 
$Dx^{(k+1)}=d^{(k)}$, con $d^{(k)}= -Cx^{(k)}+b$.

Poiché il calcolo di $x^{(k+1)}$ richiede la risoluzione del sistema $Dx^{(k+1)}=d^{(k)}$, la scelta della 
matrice $D$ deve essere fatta in modo da rendere il sistema facilmente risolvibile (e computazionalmente poco 
oneroso)
\\

Per capire se il metodo converge riscriviamo il sistema nella forma $x^{(k+1)}=-D^{-1}Cx^{(k)}+D^{-1}b$ e 
ponendo $B=-D^{-1}C=I-D^{-1}A$, consideriamo l'errore assoluto 
$e^{(k+1)}=x-x^{(k+1)}=(Bx-D^{-1}b)-(Bx^{(k)}-D^{-1}b)=B(x-x^{(k)})=Be^{(k)}$

Notiamo quindi che $e^{(k+1)}= Be^{(k)}=BB^{k}e^{(0)}=B^{k+1}e^{(0)}$, perciò perché il metodo tenda alla 
soluzione esatta vogliamo che $B^k$ tenda a zero

\paragraph{Teorema}

$\lim_{m\to\infty}B^m=O \Leftrightarrow \rho(B)<1$, dove $\rho(B)$ è il \textit{raggio spettrale} della 
matrice $B$, cioè $\max_i (|\lambda_i|)$, con $\lambda_i$ autovalori della matrice
\\

Perciò il processo iterativo è \textit{convergente} se e solo se $\boldsymbol{\rho(I-D^{-1}A)<1}$

Inoltre, per ogni norma di matrice abbiamo che $\rho(B)\leq||B||$, quindi è sufficiente che sia 
$||I-D^{-1}A||<1$

N.B. $||B||_\infty<1 \Leftrightarrow A$ è a diagonale dominante per righe
%%%%%%%%%%%%%%%%%%%%%%%%%%%%%
\subsection{Metodo di Jacobi}

N.B. se il sistema $Ax=b$ è non singolare, le sue equazioni si possono sempre riordinare in modo da avere gli 
elementi diagonali della matrice $A$ diversi da zero
\\

Scegliamo $D$ matrice diagonale, supponendo tutti i valori diagonali non nulli
\\

Il metodo di Jacobi consiste nel calcolare, nota un'approssimazione iniziale, le approssimazioni successive 
tramite la relazione
\[ \displaystyle x_i^{(k+1)}=\frac{b_i-\displaystyle\sum^n_{\substack{j=1 \\ j\neq i}}a_{ij}x_j^{(k)}}{a_{ii}} 
\qquad i\in[n] \]
%%%%%%%%%%%%%%%%%%%%%%%%%%%%%%%%%%%%
\subsection{Metodo di Gauss-Seidel}

Migliora il metodo di Jacobi introducendo le componenti già calcolate di $x^{(k+1)}$ nella sua stessa 
determinazione
\[ \displaystyle x_i^{(k+1)}=\frac{b_i-\displaystyle\sum_{j=1}^{i-1}a_{ij}x_j^{(k+1)}
-\displaystyle\sum^n_{j=i+1}a_{ij}x_j^{(k)}}{a_{ii}} \qquad i\in[n] \]

In questo caso, abbiamo scelto la matrice $D$ triangolare inferiore

I metodi di Jacobi e Gauss-Seidel sicuramente convergono quando A è una matrice a diagonale dominante, ma è 
soltanto una condizione sufficiente

Se A è simmetrica e definita positiva, il metodo di Gauss-Seidel converge

Le convergenze di questi due metodi sono indpendenti, ma quando entrambi convergono generalmente Gauss-Seidel 
è "il doppio più veloce"
%%%%%%%%%%%%%%%%%%%%%%%
\subsection{Metodo SOR}

Consideriamo dal metodo di Gauss-Seidel la differenza tra due approssimazioni della soluzione: 

$r_i^{(k)}=x_i^{(k+1)}-x_i^{(k)}=\frac{1}{a_{ii}}\left( b_i-\sum_{j=1}^{i-1}a_{ij}x_j^{(k+1)} 
- \sum_{\boldsymbol{j=i}}^na_{ij}x_j^{(k)} \right) \qquad i\in[n]$

Dove $r_i^{(k)}$ rappresenta la correzione da apportare a $x_i^{(k)}$ per ottenere la nuova approssimazione

Immaginiamo di introdurre un parametro $\omega$ tale da accelerare il più possibile la convergenza della 
successione $\{x^{(k)}\}$:

$x_i^{(k+1)}=x_i^{(k)}+\omega r_i^{(k)}$

In questo caso, il sistema è come se fosse scritto nella forma $\omega Ax=\omega b$ con decomposizione 
$\omega A = D_\omega + C_\omega$

Si nota che quando $\omega=1$ il metodo coincide con quello di Gauss-Seidel

Il calcolo del valore ottimale di $\omega$ non è in generale semplice, ma la convergenza è assicurata per 
$0<\omega<2$ se $A$ è simmetrica definita positiva
%%%%%%%%%%%%%%%%%%%%%%%%%%%%%%%%%%%%%%%%%%%
\subsection{Metodo del gradiente coniugato}

Osserviamo che la soluzione del sistema $Ax=b$ (con $A$ simmetrica e definita positiva) è equivalente alla 
minimizzazione del funzionale $\Phi(x)=\frac{1}{2}x^TAx-x^Tb$, infatti $\nabla\Phi(x)=Ax-b$ e il minimo è 
unico e coincidente con $x=A^{-1}b$
\\

Per minimizzare, seguiamo localmente la direzione di massima pendenza $-\nabla\Phi$

Se $-\nabla\Phi(x_k)=o$, allora $x_k$ è la soluzione, altrimenti, definito $r_k=b-Ax_k=-\nabla\Phi(x_k)$, 
esiste sicuramente $\alpha>0$ tale che $\Phi(x_k+\alpha r_k)<\Phi(x_k)$ (in quanto $x_k$ non è il minimo)

Sostituendo otteniamo $x_{k+1}=x_k+\alpha r_k=x_k-\alpha\nabla\Phi(x_k)$

Scegliamo $\alpha$ in modo che $\Phi(x_{(k+1)}$ sia minimo e sappiamo che il valore ottimo di $\alpha$ è 
$\alpha_k =\frac{r_k^Tr_k}{r_k^TAr_k}>0$ e si dimostra che $r^T_{(k+1)}r_k=0$, cioè i due residui sono 
ortogonali
\\

La convergenza del metodo è assicurata, ma la sua velocità diminuisce all'aumentare del \textit{numero di 
condizionamento spettrale} $K_2(A)=\frac{\lambda_1(A)}{\lambda_n(A)}$, dove 
$\lambda_1\geq\lambda_2\geq...\geq\lambda_n$ sono gli autovalori di $A$ (reali e positivi dato che $A$ è per 
ipotesi simmetrica e definita positiva)

Inoltre, si dimostra che in aritmetica con precisione infinita il metodo arriva alla soluzione esatta in al 
più $n$ iterazioni. Tuttavia, con una precisione finita di calcolo i valori nei vari step saranno approssimati 
e perciò dopo $n$ passaggi si giungerà soltanto a una approssimazione della soluzione, non necessariamente 
con la precisione di macchina. Oltre a ciò, anche se la precisione di macchina venisse raggiunta, quando la 
dimensione di $A$ è molto grande tale numero sarebbe comunque eccessivo

Il metodo converge rapidamente quando $A=I+B$ con $\text{rank}(B)\ll n$ oppure quando $K_2(A)\approx 1$. 
Se non ci si trova in queste condizioni, esistono tecniche di \textit{precondizionamento} che consentono di 
trasformare un sistema simmetrico definito positivo in uno equivalente dalla cui soluzione è possibile 
dedurre quella del sistema originale 
\\

La struttura di base del precondizionamento è, presa una matrice simmetrica non singolare 
$C$: $C^{-1}AC^{-1}Cx=C^{-1}b$\quad e ponendo $\overline{A}=C^{-1}AC^{-1} \quad \overline{x}=Cx 
\quad \overline{b}=C^{-1}b$\quad otteniamo $\overline{A}\overline{x}=\overline{b}$

Si nota che $\overline{A}$ è ancora simmetrica definita positiva, quindi si sceglie $C$ in modo da rendere 
$K_2(\overline{A})\approx 1$

È possibile scrivere l'algoritmo in modo da ottenere direttamente $x$ senza dover mai calcolare o usare 
$C^{-1}$. Si osserva che l'algoritmo dipende soltanto da $M=C^2$, simmetrica e definita positiva, che viene 
chiamata \textit{precondizionatore}. $C$ deve essere tale da rendere $Mz=r$ di facile soluzione 
\\

Uno dei precondizionatori più importanti discende dalla \textit{fattorizzazione incompleta di Choleski} della 
matrice $A$. Ha la forma $M=\overline{L}_1\overline{L}_1^T$, dove
$\left\{ \begin{array}{ll}(\overline{L}_1)_{ij}=(L_1)_{ij} & \text{se } (A)_{ij}\neq 0 
\\ (\overline{L}_1)_{ij}=0 & \text{se } (A)_{ij}=0\end{array} \right.$ con $L_1$ matrice triangolare 
inferiore della decomposizione di Choleski classica



\chapter{Approssimazione di funzioni}

\end{document}





































